\documentclass[a4paper,10pt]{report}
\usepackage[utf8]{inputenc}
\usepackage[pdftex]{graphicx}
\usepackage{fancyhdr}
\usepackage{listings}
\usepackage[francais]{babel}
\usepackage{hyperref}
\usepackage{amsmath}

% Title Page
\title{IA01 : TP2}
\author{Par Camille Gerin-Roze et Thomas Perrin}



\renewcommand{\footrulewidth}{1pt}
\fancyfoot[C]{\textbf{\thepage}}
\fancyfoot[L]{}

\hypersetup{
	colorlinks=true,
	linkcolor=black
}

\begin{document}

\begin{titlepage}

\begin{flushright}
  \includegraphics[scale = 0.2]{logo_utc.jpg}
\end{flushright}
\vspace*{5cm}

\newcommand{\HRule}{\rule{\linewidth}{0.5mm}} % Defines a new command for the horizontal lines, change thickness here
\center % Center everything on the page



%----------------------------------------------------------------------------------------
%	TITLE SECTION
%----------------------------------------------------------------------------------------

\HRule \\[0.4cm]
{ \LARGE \bfseries  IA01 : Rapport du TP3}\\[0.4cm] % Title of your document
\HRule \\[1.5cm]

%----------------------------------------------------------------------------------------
%	AUTHOR SECTION
%----------------------------------------------------------------------------------------
\begin{minipage}{0.4\textwidth}
\begin{flushleft} \large
\emph{Auteurs:}\\
Camille \textsc{Gerin-Roze} \newline
Thomas \textsc{Perrin}
\end{flushleft}
\end{minipage}
~
\begin{minipage}{0.4\textwidth}

\end{minipage}\\[1.3cm]


{\large Le 3 Janvier 2016} % Date, change the \today to a set date if you want to be precise

\end{titlepage}
\tableofcontents
\chapter*{Introduction}

Dans tous univers fantaisistes, les nains sont un peuple vivant dans les mines, et ceux-ci sont toujours amené à creuser des tunnels. Il faut donc, pour les différents chantiers, pouvoir déterminer le nombres de nains que l’ont doit mettre sur ces derniers. Nous allons nous baser sur cette vidéo : 
\newline \url{https://www.youtube.com/watch?v=C75iVDrfy2w}. 
La conférence fut dirigé par John Lang, créateur du Donjon de Naheulbeuk.

Cette vidéo répond à la question : “Combien faut-il de nain pour creuser en 2 jours un tunnel de 28 mètre dans du granite ?”. Notre système expert va permettre de généraliser le problème à plusieurs types de chantier en faisant varier la longueur, largeur et type du tunnel, mais aussi la durée du chantier. 

Ainsi, en fonction des paramètres donnés, on va déterminer si le délais peut être respecté, avec la présence ou non d’une équipe de nuit pour aboutir sur un nombre total de nains. 

Par exemple, on sait qu’il faut 8 nains de ravitaillement de bière pour deux autres nains, puisque ces derniers sont proches des tonneaux de bières et deviennent ivres rapidement. De même, les nains n’appréciant pas de boire de la bière dans des chopes sales, on rajoute un plongeur pour 4 nains de ravitaillement de bière. 


\chapter{Connaissances nécessaires au système expert}
 \section{Paramètres à fournir au système}
 Afin de déterminer le nombre de nains nécessaire à un chantier, plusieurs paramètres seront nécessaire :
\begin{itemize}
\item La longueur du tunnel
\item Sa hauteur
\item Sa largeur
\item Le nombre de jours maximum
\item Le type de roche
\item Le type de pioche employé
\end{itemize}

Avec ces paramètres, on va pouvoir déterminer si oui ou non le chantier est réalisable. Si il l’est, on obtiendra un nombre total de nains avec les différents rôles de chacun. Un indicateur sera présent pour dire si une équipe de nuit est nécessaire ou non pour compléter le chantier en temps et en heures.

\section{Traduction des connaissances en règles}

La base de règles est décomposé en plusieurs parties :
\begin{itemize}
\item Les règles du type de roche (RTR)
Ces règles vont permettre, en fonction du type de roche, de fixer la vitesse de travail d’un nain par jour.

\item Les règles du type de pioche (RTP)
Ces règles vont ajouter un coefficient multiplicateur à la vitesse de base, pour aller plus ou moins vite en fonction des moyens de l’employeur.

\item Les règles concernant la réalisation du chantier (RCR)
Ces règles vont déterminer si le chantier est réalisable avec les paramètres énoncés. En effet, en fonction de la vitesse des nains, de la hauteur, largeur et longueur du tunnel, on peut d’avance vérifier si cela tiendra avec le nombre de jours énoncé ou non.

\item Les règles calculant le nombre de nains (RN)
Ces dernières règles vont conclure en calculant le nombre de nains totals.
\end{itemize}

\section{Origine des règles}

L'idée de ce système expert nous est venu par la vidéo présenté ci-dessus. John Lang, le créateur de l’univers du Donjon de Naheulbeuk nous a également aidé en tant qu'expert dans ce domaine. 
Les RN sont issus de la vidéo présentant un cas pratique. La première règle va créer l'équipe qui sera présente dans tous les cas, c'est-à-dire l'équipe de jour. Les deux suivantes sont en fonction de la présence d'une équipe de nuit ou non : si une équipe de nuit est présente, on double l'effectif de jour pour qu'il soit présent pendant la nuit, puis on ajoute les nains spécifiques à la nuit et les nains de ravitaillements et plongueurs. Autrement, on complète juste l'équipe de jour avec les nains de ravitaillements et plongueurs correspondant.
Nous avons seulement modifié une seule règle pour la rendre cohérente, celle des nains de ravitaillements en bière. En effet, lorsque John Lang ajoute plus de nains de ravitaillement quand l'effectif augmente, les proportions ne sont pas exactement respectés. Nous avons donc corrigé cela.
Les RCR sont des calculs mathématiques, de la logique uniquement.
Les RTP sont issus de notre échange avec John Lang qui nous a indiqué que différentes pioches pourraient influencés la vitesse de chaque nain.
Enfin, les RTR sont issus également de notre échange avec John Lang. Nous avons extrapôlé les différents types de roches possibles en se renseignant sur les différents types de minerai.
\newpage
\section{Arbre de déduction}

\begin{figure}[!htp]
  \centerline{\includegraphics[scale=1]{arbre_inference}}
\end{figure}


\chapter{Implémentation du système expert}
  \section{Choix de la représentation Lisp}
    \subsection{Base de règles}
    Pour représenter les règles de notre SE, nous avons choisi d'utiliser une liste qui contiendra 2 sous-listes ( les nouveaux faits et les prémisses )
    ainsi que le nom de la règle. Par exemple, voici la règle RCR1 qui permet de savoir si un chantier est réalisable:\newline
    \begin{lstlisting}[language=Lisp]
(
	( ;;;; DEBUT DES NOUVEAUX FAITS
		(ChantierRealisable . T)
		(EquipeDeNuit . NIL)
	);;;; FIN DES NOUVEAUX FAITS
		(;;;;DEBUT DES PREMISSES
		(COMPARAISON (>=
			(* LargeurTunnel HauteurTunnel
			    NombreDeJours VitesseNain)
			LongueurTunnel
		))
	);;;;FIN DES PREMISSES
RCR1);;;;NOM DE LA REGLE

    \end{lstlisting}

    On peut noter que les premisses ont un format bien spécial. En effet, chaque condition commence par le type de vérification à faire. Cela peut être une égalité,
    une comparaison, ou encore une définition (savoir si un fait a déjà été défini). Cette étiquette permet de simplifier le traitement lors de la vérification
    des prémisses. Pour les prémisses, une fonction va donc vérifier si LargeurTunnel * HauteurTunnel * NombreDeJours * VitesseNain est bien supérieur ou égale à la
    longueur du tunnel. Si c'est le cas, notre moteur d'inférence va apeller une nouvelle fonction qui va évaluer les valeurs des nouveaux faits et les ajouter à
    notre base de fait. Nous avons également une A-List qui contient la description de chaque règles, mais celle-ci a été mise à part dans le but de garder un code
    clair. Cette représentation permet non seulement d'avoir un format homogène pour toutes les règles, et donc de traiter chaque règle de la même façon, mais elle
    permet également de traiter des conditions complexes qui demandent l'évaluation de plusieurs opérations ou encore des opérations qui demandent de rechercher
    dans la base de faits.

    \subsection{Base de faits}
      La base de faits a été faites de la manière la plus simple possible, celle-ci est juste une liste dont chaque membre est une sous-liste qui représente un fait.
      Un fait est tout simplement constitué d'une étiquette, donc le nom du fait, et d'une valeur qui peut être un nombre, un symbole, ou même une valeur booléenne.
      \begin{lstlisting}[language=Lisp]
(
  (VitesseNain 3)
  (TypeRoche GRANITE)
  (LargeurTunnel 3)
  (LongueurTunnel 28)
  (HauteurTunnel 1)
)
      \end{lstlisting}
      Ici on peut voir qu'il y a 4 faits dans notre base de faits : la vitesse des nains, le type de roche, la largeur et la longueur du tunnel, et la hauteur du tunnel.
  \section{Le moteur d'inférences}
  Le moteur d'inférence est un moteur à chaînage avant avec un fonctionnement simple, mais pour bien comprendre comment il fonctionne, il faut expliciter certaines
  fonctions qui permettent de savoir si les prémisses sont satisfaits ou bien d'ajouter les faits à notre base de faits. Nous allons donc commencer par présenter
  toutes ces fonctions.\newpage
    \subsection{Validation des prémisses}
    Pour que les prémisses soient validés, il faut que toutes les conditions soient validées. Donc nous allons itérer sur toutes les conditions de la liste
    des prémisses pour vérifier qu'elles sont toutes validée. Mais certaines conditions sont complexes et demandent d'aller vérifier l'existence ou la valeur
    de faits dans la base de faits. Pour cela nous avons défini une fonction evalOperande qui va nous permettre d'évaluer la valeur de chaque opérande :
    \begin{lstlisting}[language=Lisp]
(defun evalOperande(operande)
  (let ((newList (list)))
      (dolist (x operande newList)
        (cond
          ((LISTP x)
            (setq newList (append newList (list (evalOperande x)))))
          ((OR (EQUAL x '*) (EQUAL x '/) (EQUAL x '-) (EQUAL x '+) (EQUAL x 'truncate))
            (setq newList (APPEND newList (list x))))
          ((SYMBOLP x)
            (if (NULL (ASSOC x *BaseFaits*))
	      ;;;; SI LE SYMBOLE N'EST PAS DEFINI DANS LA BDF
                (return-from evalOperande NIL) 
              (setq newList (APPEND newList (cdr (ASSOC x *BaseFaits*))))
            ))
          (T (setq newList (APPEND newList (list x))))
        )
    )
  )
)    
    \end{lstlisting}

    Pour évaluer un opérande, on va vérifier si celui-ci est une liste, si c'est une liste on va alors appeller notre fonction récursivement sur chaque éléments
    de la liste. Dans le Cas où c'est un symbole, on va alors aller chercher sa valeur dans la base de faits. Si jamais la valeur n'est pas dans la base de faits,
    il est important que la fonction renvoie NIL pour qu'on sache que cela ne peut pas être évalué et que la condition n'est pas respectée. À la fin de la fonction,
    on aura une liste qui pourra être évaluée grâce à un simple eval.\newline
    Ensuite il y a deux autres fonctions, une qui va vérifier si une condition est respectée, et une qui vérifiera si toutes les conditions sont vérifiées. La fonction
    qui vérifie si une condition est respectée va d'abbord vérifier l'étiquette placée au début de celle-ci (COMPARAISON, DEFINI, EGALITE) pour savoir comment elle 
    doit la traiter.\newpage
		\begin{lstlisting}[basicstyle = \footnotesize, language=Lisp]
(defun conditionRespecte?(condition)
(cond
  ((EQUAL (car condition) 'EGALITE)
      (return-from conditionRespecte? 
	  (EQUAL (cadr (ASSOC (car (cadr condition)) *BaseFaits*)) 
	   (cadr (cadr condition)))))
  ((EQUAL (car condition) 'DEFINI)
      (return-from conditionRespecte? 
	  (NOT (NULL (ASSOC (car (cadr condition)) *BaseFaits*)))))
  ((EQUAL (car condition) 'COMPARAISON)
    (return-from conditionRespecte?
      (eval
        (let ((expression))
          (dolist (x (cadr condition) expression)
            (cond
              ((estUnComparateur? x)
                (setq expression (append expression (list x))))
                ((listp x)
                  (if (evalOperande x)
                    (setq expression 
		      (append expression (list (eval (evalOperande x))))
		    )
                      NIL
                    ))
                (T
                  (if (ASSOC x *BaseFaits*)
                    (setq expression 
		      (append expression (list (cadr (ASSOC x *BaseFaits*))))
		     )
                    NIL
                    ))
            )
          )
        ))))
  )
)		


(defun premisseRespecte?(regle)
  (let ((test T))
    (dolist ( x (getPremisse regle) test)
      (setq test (AND test (conditionRespecte? x)))
    )
  )
)
		\end{lstlisting}

Un fois que la condition est vérifiée, on peut tout simplement faire une boucle sur toutes les conditions pour savoir si celles-ci sont toutes vrais, si elles sont 
toutes vrais on pourra ajouter les nouveaux faits à notre base de fait.\newpage
    \subsection{Calcul des nouvelles valeurs}
    Dans notre système expert, il est possible qu'une variable soit mises à jour dans certains conditions. Il est aussi possible qu'il faille calculer la valeur 
    de la variable de notre système expert, donc l'ajout d'un nouveau fait n'est pas si aisé. 
    \begin{lstlisting}[basicstyle = \footnotesize, language=Lisp]
(defun ajouterFait(fait)
  (if (assoc (car fait) *BaseFaits*)
    (setq *BaseFaits* (remove (car fait) *BaseFaits* :key #'first)))
    (push fait *BaseFaits*)
)

(defun evaluerValeur(val)
  (cond
    ((OR (EQUAL val T) (NULL val)(NUMBERP val)) val)
    ((SYMBOLP val) (cadr (ASSOC val *BaseFaits*)))
    ((LISTP val)
      (let (newVal)
          (dolist (x val newVal)
            (if (OR 
		  (EQUAL x '+) (EQUAL x '-) (EQUAL x '/) 
		  (EQUAL x '*) (EQUAL x 'truncate) (EQUAL x 'ceiling)
		)
              (setq newVal (APPEND newVal (list x)))
              (setq newVal (APPEND newVal (list (evaluerValeur x))))
            )
          )
        )
    )
  )
)

(defun majBDF(regle)
  (dolist (x (getNouveauxFaits regle) NIL)
    (if (listp (evaluerValeur (cdr x)))
      (ajouterFait (list (car x) (eval (evaluerValeur (cdr x)))))
      (ajouterFait (list (car x) (cdr x)))
    )
  )
)

    \end{lstlisting}

    La fonction evaluerValeur va couvrir plusieurs cas : 
    \begin{itemize}
     \item Dans le cas où la valeur passée en paramètre est un nombre ou une valeur logique ( T ou NIL ) alors on ne fait rien
     \item Dans le cas où la valeur est un symbole, on va aller chercher sa valeur dans la base de faits
     \item Et dans le cas où la valeur est une liste, alors on va créer une nouvelle expression évaluable en allant chercher toutes les valeurs dont on a besoin dans la base de faits.
    \end{itemize}
    
    En suite, la mise à jour de la base de faits se fait en évaluant notre nouvelle expression dans le cas où elle a besoin d'être évaluée.

    \newpage
    \subsection{Moteur à chaînage avant}
    
    \begin{lstlisting}[basicstyle = \footnotesize, language=Lisp]
(defun chainageAvant()
  (let ((listeRegle *BaseRegles*)(nouveauxFaits T)(reglesUtilise NIL))
  (loop while nouveauxFaits do
    (setq nouveauxFaits NIL)
    (dolist (r listeRegle NIL) ;;;; TANT QU'IL Y A DES RÈGLES À TESTER
          (format t "~%REGLE=~a~%BDF=~a" (caddr r) *BaseFaits*)
            (if (premisseRespecte? r) ;;;; SI LES PREMISSES DE LA REGLE SONT RESPECTES
                (progn
                    (setq nouveauxFaits T) ;;;; ON NOTIFIE QU'IL Y A EUT UN NOUVEAU FAIT
                    
                    ;;;; ON REITIRE LA REGLE DE LA LISTE
                    (setq listeRegle (remove (caddr r) listeRegle :key #'third))  
                    (majBDF r)
                    (push (caddr r) reglesUtilise)
                )
            )
      )
  )
  reglesUtilise
)
)
    \end{lstlisting}

    Le moteur d'inférence est un moteur d'inférence en chaînage avant et en profondeur d'abbord car on ajoute les faits directement après qu'ils aient été déterminés
    par une règle. On notera également que le moteur de chainage avant va renvoyer la liste des règles utilisées, mais la base de fait étant une variable globale 
    celle-ci est directement mise à jour.
    \newpage
    \subsection{Test du moteur d'inférences}
    
    Pour pouvoir utiliser notre système expert nous avons créé une interface utilisateur, voici un exemple d'utilisation du système expert avec les données du problème
    exposé par John Lang. On peut remarquer que le résultat est légèrement différent, ceci est dû au fait qu'en correspondant avec lui, nous avons eu de nouvelles 
    informations qui nous ont permis de d'enrichir le système expert, d'où la petite différence de résultat.
    
    \begin{lstlisting}[basicstyle = \footnotesize]
    
De quel longueur est votre tunnel ?28
De quel largeur est votre tunnel ?4
Quel est la hauteur de votre tunnel ?1
Dans combien de temps votre tunnel doit être prêt ?2
Selectionner le type de roche dans lequel vous voulez creuser : GRANITE

... DEROULEMENT DE TOUTES LES REGLES ...

--------------------------------------------------------------------
--------------     LE CHANTIER EST REALISABLE     ------------------
--------------------------------------------------------------------

VOILÀ LA COMPOSITION DE VOS EQUIPES POUR FAIRE VOTRE CHANTIER LE 
PLUS RAPIDEMENT POSSIBLE : 

VOUS AVEZ BESOIN D'UNE EQUIPE DE JOUR COMPOSÉE DE : 
Nains mineurs : 3
Nains Guerisseurs : 1
Nains Forgerons : 1
Nains Tourneurs de manche : 1
Nains Ravitailleurs : 12
Nains Plongeurs (Vaisselle) : 3

AINSI QUE D'UNE EQUPIDE DE NUIT COMPOSÉE DE : 
Nains Mineurs : 3
Nains Guerisseurs : 1
Nains Forgerons : 1
Nains Tourneurs de manche : 1
Nains Ravitailleurs : 12
Nains Plongeurs (Vaisselle) : 3
Nains Surveillants : 4
Nains Managers : 2
Nains Porteurs de lanternes : 4

TOTAL :                                                52 nains
--------------------------------------------------------------------
(RN3 RN1 RCR2 RTP4 RTR2)
    \end{lstlisting}

\chapter{Scénarios d'utilisation}

  \section{Chantier non réalisable}
  
    Les nains n'apprécient pas qu'on leur marche sur les pieds, donc il faut faire attention de ne pas mettre trop de nain dans un tunnel qui n'est pas assez large.
    On va faire tourner notre moteur d'inférence avec les données suivantes : 
    \begin{itemize}
     \item Largeur du tunnel : 2
     \item Longueur du tunnel : 100
     \item Hauteur du tunnel : 1
     \item Temps maximale : 2
     \item Type de roche : 
    \end{itemize}
    
    Avec ces données, on a le résultat suivant:
    \begin{lstlisting}
BDF = 
((EQUIPEDENUIT NIL) (CHANTIERRÉALISABLE NIL) (VITESSENAIN 3)
(TYPEDEPIOCHE STANDARD) (TYPEDEROCHE GABBROS) (HAUTEURTUNNEL 1)
(NOMBREDEJOURS 2) (LONGUEURTUNNEL 100) (LARGEURTUNNEL 2))
--------------------------------------------------------------------
-----------     LE CHANTIER EST N'EST PAS REALISABLE     -----------
--------------------------------------------------------------------


(RCR3 RTP4 RTR3)

    \end{lstlisting}

    On remarque que la vitesse des nains est de 3 à cause de la roche, Mais le chantier n'est pas réalisable. Les trois
    règles utilisées sont : 
        \begin{itemize}
	    \item RCR3 : Si $$LongueurTunnel < LargeurTunnel*hauteurTunnel*nbJourMax*VitesseNain*2$$ alors le chantier n'est pas réalisable
	    \item RTP4 : Si la pioche est standard, alors la vitesse du nain est multipliée par 1
            \item RTR3 : Si la roche est de type Gabbros, alors le nain avancera avec une vitesse de 3.
	\end{itemize}
  
    Donc il est tout à fait normal que le chantier soit irréalisable, car il faudrait que le tunnel soit beaucoup plus large pour pouvoir y mettre plus de nains.
  \section{Chantier réalisable sans équipe de nuit}
  
  Pour le chantier réalisable sans équipe de nuit il faut un tunnel assez large pour pouvoir accueuillir beaucoup de nain mineurs, donc nous allons tester avec les
  valeurs suivantes : 
  \begin{itemize}
     \item Largeur du tunnel : 6
     \item Longueur du tunnel : 28
     \item Hauteur du tunnel : 1
     \item Temps maximale : 2
     \item Type de roche : Granite
    \end{itemize} 
    
  \begin{lstlisting}[basicstyle = \footnotesize]
BDF = 
((NAINSCALCULÉ T) (NBNAINTOTAL 42) (NBNAINPLONGUEUR 6)
 (NBNAINRAVITAILLEMENT 24) (NBNAINTOURNEURMANCHE 2) (NBNAINFORGERON 2)
 (NBNAINGUERISSEUR 2) (NBNAINMINIER 6) (EQUIPEDENUIT NIL)
 (CHANTIERRÉALISABLE T) (VITESSENAIN 3) (TYPEDEROCHE GRANITE) (HAUTEURTUNNEL 1)
 (NOMBREDEJOURS 2) (LONGUEURTUNNEL 28) (LARGEURTUNNEL 8) (NBNAINMANAGER 2)
 (NBNAINSURVEILLANT 4) (NBNAINPORTEURLANTERNE 4) (TYPEDEPIOCHE STANDARD))

--------------------------------------------------------------------
--------------     LE CHANTIER EST REALISABLE     ------------------
--------------------------------------------------------------------

VOILÀ LA COMPOSITION DE VOS EQUIPES POUR FAIRE VOTRE CHANTIER LE 
PLUS RAPIDEMENT POSSIBLE : 

VOUS AVEZ BESOIN D'UNE EQUIPE DE JOUR COMPOSÉE DE : 
Nains mineurs : 6
Nains Guerisseurs : 2
Nains Forgerons : 2
Nains Tourneurs de manche : 2
Nains Ravitailleurs : 24
Nains Plongeurs (Vaisselle) : 6


TOTAL :                                                42 nains

--------------------------------------------------------------------

(RN2 RN1 RCR1 RTP4 RTR2)
\end{lstlisting}

\begin{itemize}
     \item RN2 : Si il n'y a pas d'équipe de nuit et que l'équipe de jour est commencée, on complète l'équipe de jour avec les nains de ravitaillement et les nains plongueurs.
     \item RN1 : Si le chantier est réalisable, alors on crée le début de l'équipe de jour qui sera présente dans tous les cas (nain miniers, nains guérisseurs, nains tourneurs de manche et nains forgeron).
     \item RCR1 : Si LongueurTunnel inferieur à LargeurTunnel*hauteurTunnel*nbJourMax*VitesseNain, le chantier est réalisable et nous n'avons pas besoin d'équipe de nuit
     \item RTP4 : Si la pioche est standard, alors la vitesse du nain est multipliée par 1
     \item RTR2 : Si la roche est de type Granite, alors le nain avancera avec une vitesse de 3.
\end{itemize}
  
  Les résultats sont cohérents, en effet 6 nains mineurs ayant une vitesse de 3 vont pouvoir creuser 18m par jour, donc en deux jours ils auront fini le tunnel.
  
  \newpage
  \section{Chantier réalisable avec une équipe de nuit}
  Pour le chantier réalisable avec une équipe de nuit nous allons utiliser les mêmes valeurs que le problème originel et que nous avons utilisé dans le chapitre 2. 
  \begin{itemize}
     \item Largeur du tunnel : 4
     \item Longueur du tunnel : 28
     \item Hauteur du tunnel : 1
     \item Temps maximale : 2
     \item Type de roche : Granite
    \end{itemize}
  \begin{lstlisting}[basicstyle = \footnotesize]
BDF = 
((NAINSCALCULÉ T) (NBNAINTOTAL 52) (NBNAINPLONGUEUR 6)
 (NBNAINRAVITAILLEMENT 24) (NBNAINMANAGER 2) (NBNAINSURVEILLANT 4)
 (NBNAINTOURNEURMANCHE 2) (NBNAINFORGERON 2) (NBNAINGUERISSEUR 2)
 (NBNAINMINIER 6) (NBNAINPORTEURLANTERNE 4) (EQUIPEDENUIT T)
 (CHANTIERRÉALISABLE T) (VITESSENAIN 3) (TYPEDEROCHE GRANITE) (HAUTEURTUNNEL 1)
 (NOMBREDEJOURS 2) (LONGUEURTUNNEL 28) (LARGEURTUNNEL 4)
 (TYPEDEPIOCHE STANDARD))

--------------------------------------------------------------------
--------------     LE CHANTIER EST REALISABLE     ------------------
--------------------------------------------------------------------

VOILÀ LA COMPOSITION DE VOS EQUIPES POUR FAIRE VOTRE CHANTIER LE 
PLUS RAPIDEMENT POSSIBLE : 

VOUS AVEZ BESOIN D'UNE EQUIPE DE JOUR COMPOSÉE DE : 
Nains mineurs : 3
Nains Guerisseurs : 1
Nains Forgerons : 1
Nains Tourneurs de manche : 1
Nains Ravitailleurs : 12
Nains Plongeurs (Vaisselle) : 3

AINSI QUE D'UNE EQUPIDE DE NUIT COMPOSÉE DE : 
Nains Mineurs : 3
Nains Guerisseurs : 1
Nains Forgerons : 1
Nains Tourneurs de manche : 1
Nains Ravitailleurs : 12
Nains Plongeurs (Vaisselle) : 3
Nains Surveillants : 4
Nains Managers : 2
Nains Porteurs de lanternes : 4

TOTAL :                                                52 nains

--------------------------------------------------------------------
(RN3 RN1 RCR2 RTP4 RTR2)
 \end{lstlisting}
 \begin{itemize}
      \item RN3 : Si une équipe de nuit est nécessaire et que l'équipe de jour est commencé, 
      on crée l'équipe de nuit (Nains porteur de lanterne, nains surveillant, nains manager), 
      on double l'effectif de l'équipe de jour pour avoir l'équivalence en équipe de nuit 
      et on termine par les nains de ravitaillement et les nains plongueurs. 

      \item RN1 : Si le chantier est réalisable, alors on crée le début de l'équipe de jour qui sera présente dans tous les cas 
      (nain miniers, nains guérisseurs, nains tourneurs de manche et nains forgeron).

      \item RCR2 : Si LargeurTunnel*hauteurTunnel*nbJourMax*VitesseNain <= LongueurTunnel <= LargeurTunnel*hauteurTunnel*nbJourMax*VitesseNain*2 , 
	le chantier est réalisable et nous avons besoin d'une équipe de nuit

      \item RTP4 : Si la pioche est standard, alors la vitesse du nain est multipliée par 1

      \item RTR2 : Si la roche est de type Granite, alors le nain avancera avec une vitesse de 3.
\end{itemize}
  Les résultats sont très proches de ceux exposés par John Lang dans la vidéo qui nous a permis de faire la plus grande partie des règles, mais il est normal qu'il
  y ait une légère différence car nous prenons en comptes d'autres données en compte. De plus nous avons correspondu avec John Lang, ce qui nous a forcé à modifier
  des règles. Les règles qui ont été utilisée sont cohérentes et correspondent bien aux données originels du problème (le type de pioche se modifie dans les configurations
  de l'UI et est mis à standard par défaut).
  Si 3 nains creusent jours et nuit, en deux jours et deux nuits ils auront creusé 3*4*3= 36m dans le granite, donc ils auront fini le tunnel.
   
\chapter*{Conclusion}

  Le système expert que nous avons réalisé nous a permis de comprendre les mécanismes d'inférence d'un système expert. Nous avons également pu nous rendre compte
  qu'il est très important de bien structurer ses données de manière à pouvoir les utiliser facilement. Les échanges avec John Lang ont confirmé la vision que nous avions du calcul du nombre du nains tout en ajoutant quelques paramètres supplémentaire tels que le type de tunnel et le type de pioche. Ainsi, cela confirme l'importance d'avoir un expert du domaine lors d'une application concrète.
  Ce système expert pourrait être amélioré en ajoutant 
  le calcul du coup d'un nain ou encore l'ajout d'autre paramètres pour avoir un résultat plus précis.
  
\appendix
\chapter{Mail de John Lang}
Salutations !

Ah, bah on dirait bien que vous êtes encore plus timbrés que moi, les amis :)
C'est sans doute une bonne chose !\newline


- Quand vous parlez de Tunnel standard, combien mesure en largeur ce tunnel ? Et donc combien faut il de mètre de largeur par nain minier pour que ceux-ci ne se bagarrent pas ? 
\newline
Un tunnel standard fait 2,5 m de large, soit l'épaisseur de deux nains en armure. On doit pouvoir y poser des rails de mine et pouvoir circuler sur un côté. Il faut donc deux nains de front pour le creuser, mais si le tunnel est haut, il peut y en avoir quatre à cause de l'échafaudage. Ils ne se battront que si on oublie de leur livrer leur bière.
\newline

- Quels sont les différents types de pioches pouvant être utilisés par les nains ? Et dans quels mesures l'utilisation de celles-ci permet au nain de creuser plus vite ? 
\newline

On a parlé de la pioche double et de la pioche simple, pour le moment. Je pense que l'utilisation de matériaux de haut niveau (tel que le Thritil) pourrait augmenter la puissance de la pioche et le rendement du mineur, mais pour un prix d'outil environ 6 fois supérieur.
\newline

-Peut on supposer qu'en fonction du type de roche les nains iront plus ou moins vite ? \newline

Alors ça, c'est certain... Le granite c'est hyper dur à casser, contrairement à d'autres roches comme le grès ou le calcaire. Là je pense qu'il vous faudra faire appel à un géologue de métier pour faire obtenir une classification exacte ;)

Continuez ce que vous faites :)

Merci


  \end{document}
